%!TeX TS-program = Lualatex 
%!TeX encoding = UTF-8 Unicode 
%!TeX spellcheck = en-US

\documentclass[stdletter,8pt,dateno]{newlfm}%
%\usepackage{kpfonts}
\usepackage{etoolbox}

\makeatletter
\patchcmd{\@zfancyhead}{\fancy@reset}{\f@nch@reset}{}{}
\patchcmd{\@set@em@up}{\f@ncyolh}{\f@nch@olh}{}{}
\patchcmd{\@set@em@up}{\f@ncyolh}{\f@nch@olh}{}{}
\patchcmd{\@set@em@up}{\f@ncyorh}{\f@nch@orh}{}{}
\makeatother

\usepackage{url}
%: METADATA
%: %%%%%%%%%%%%%%%%%%%%%%%%%%%%%%%%%%%%%%%%%%%%%%%%%%%%%%%%%%%%%%%%%%%%
\newcommand{\AuthorA}{Chlo\'e Pasturel}
\newcommand{\AuthorB}{Anna Montagnini}%
\newcommand{\AuthorC}{Laurent Perrinet}%
\newcommand{\Address}{Institut de Neurosciences de la Timone, CNRS / Aix Marseille Univ - Marseille, France}%
\newcommand{\Website}{https://laurentperrinet.github.io}%
\newcommand{\EmailC}{Laurent.Perrinet@univ-amu.fr}%
\newcommand{\Title}{
%Principles and psychophysics of Active Inference in anticipating a dynamic probabilistic bias
% Should I stay or should I go? 
%Humans adapt their eye movements to the volatility of visual motion properties, and know about it
Humans adapt their anticipatory eye movements to the volatility of visual motion properties
%Anticipating a volatile probabilistic bias in visual motion direction
%Humans adapt to the volatility of visual motion properties :  eye movements and explicit guesses
}
\newcommand{\Acknowledgments}{This work was supported by EU Marie-Sklodowska-Curie Grant No 642961 (PACE-ITN) and by the Fondation pour le Recherche M\'edicale, under the program \textit{Equipe FRM} (DEQ20180339203/PredictEye/G Masson). Code and material on the \href{\Website/publication/pasturel-montagnini-perrinet-19}{corresponding author's website}. We thank Doctor Jean-Bernard Damasse, Guillaume S Masson and Professor Laurent Madelain for insightful discussions. }
\newcommand{\Abstract}{
Animal behavior must constantly adapt to changes, for example when the statistical properties of the environment change unexpectedly. For an agent that interacts with this volatile setting, it is important to react accurately and as quickly as possible. It has already been shown that when a random sequence of motion ramps of a visual target is biased to one direction (e.g. right or left), human observers adapt to accurately anticipate the expected direction with their eye movements. Here, we prove that this ability extends to a volatile environment where the probability bias could change at random switching times. In addition, we also recorded the explicit direction prediction reported by observers as given by a rating scale. Both results were compared to the estimates of a probabilistic agent that is optimal in relation to the event switching generating model. Compared to the classical leaky integrator model, we found a better match between our probabilistic agent and the behavioral responses, both for the anticipatory eye movements and the explicit task. Furthermore, by titrating the level of preference between exploration and exploitation in the model, we were able to fit each individual experimental data-set with different levels of estimated volatility and derive a common marker for the inter-individual variability of participants. These results prove that in such an unstable environment, human observers can still represent an internal belief about the environmental contingencies, and use this representation both for sensory-motor control and for explicit judgments. This work offers an innovative approach to more generically test the diversity of human cognitive abilities in uncertain and dynamic environments.}
%%%%%%%%%%%%%%%%%%%%%%%%%%%%%%%%%%%%%%%%%%
%\newcommand{\Journal}{Journal of Vision}%
\newcommand{\Journal}{PLoS Computational Biology}%
%\newcommand{\Journal}{eLife}%
\widowpenalty=1000
\clubpenalty=1000

\newsavebox{\Lpalmb} \sbox{\Lpalmb}{\parbox[t]{1.75in}{\includegraphics[width=1.\textwidth]{/Users/lolo/nextcloud/libs/slides.py/figures/troislogos.png}}}
%\makelth{Homea}{\Lheader{\usebox{\Lpalms}}}%
%
\Lheader{\usebox{\Lpalmb}}

\newlfmP{headermarginskip=2pt}
\newlfmP{sigsize=2pt}
\newlfmP{dateskipafter=2pt}
%\newlfmP{addrfromphone}
\newlfmP{addrfromemail}
%\PhrPhone{Phone}
\PhrEmail{Email}

\namefrom{\AuthorA , \AuthorB\ and \AuthorC\ }
\addrfrom{%
\AuthorC\ \\
%\Address \\
%\AuthorB\\[6pt]
    \Address\\[6pt]
%\AuthorA\\[6pt]
%%    \AddressA
%    \LongAddressA
}
%        \phonefrom{\PhoneA}
\emailfrom{\EmailC\\[6pt]
    }

\addrto{%
    \today
}

\greetto{
To the editorial board of \emph{\Journal},%
%To Whom It May Concern,
%Dear XXX
}
\closeline{Sincerely,}

\begin{document}
\begin{newlfm}
%Cover Letter:
%
%    How will your work make others in the field think differently and move the field forward?
%    How does your work relate to the current literature on the topic?
%    Who do you consider to be the most relevant audience for this work?
%    Have you made clear in the letter what the work has and has not achieved?
Please consider this submission for publication in \Journal . This inter-displinary work studies how ``\emph{\Title}'' by defining a theoretically-driven experimental protocol which is then validated in different behavioral experiments.

Understanding how humans adapt to changing environments to make judgments or plan motor responses based on time-varying sensory information 
is indeed crucial for psychology, neuroscience
and artificial intelligence.  
Current theories for how we deal with the environment's uncertainty most rely on \emph{equilibrium} behavioral responses in experiments where some stationary randomness is introduced.  
Here we show that in the case where the context switches at random times during the experiment, an adaptation to the \emph{volatility} can be performed online.
In particular, we show in two behavioral experiments that humans can adapt to such volatility at the early sensorimotor level, through their anticipatory eye movements, but also at a higher cognitive level, through explicit ratings. 
Our results suggest that humans (and future artificial
systems) can use much richer adaptive strategies than previously
assumed.

Importantly, we provide in this work a full theoretical hierarchical Bayesian model for the study, which allows the rigorous analysis of the behavioral data and which we think will provide an essential tool to move forward the field. 
As such, we believe that this work presents a highly novel and significant finding of broad interest 
which will have an influential impact on the readership of \Journal .

%The cover letter should state clearly what is included as the submission, including number of words in the text and number of display items (figures, tables, boxes) in the print version of the paper; number of additional words in the text (full Methods and Extended Data legends) and number of Extended Data figures and tables for the online-only version; any Supplementary Information (specifying number of items and format); number of supporting manuscripts.
Our submission has a 262-word summary and a body of 19178 words
(including methods, figure legends and appendix), along with five figures.
% pdftotext Pasturel_etal2019.pdf - | wc -w

The authors declare no competing interests.
We did not have any prior discussions with
a \Journal\ Editorial Board Member
about the work described in the manuscript.

Thank you for your consideration, and we look forward to your response.

\end{newlfm}
\end{document}
